
\documentclass{article}
\usepackage{palatino}
\usepackage[utf8]{inputenc}
\usepackage[swedish]{babel}
\usepackage[T1]{fontenc}
\usepackage{graphics}
\usepackage{graphicx}
\usepackage{wrapfig}
\usepackage[pdftex,colorlinks=true,urlcolor=blue,pdfstartview=FitH]{hyperref}
\nonfrenchspacing
\clubpenalty=9999
\widowpenalty=9999 
\setcounter{tocdepth}{3}

\title{Kompilera program från källkod i SUSE Linux}
\author{Jonas Björk\\ \small{jonas@jonasbjork.net}}
\date{2008-01-12}

\begin{document}
\maketitle
\setlength{\parindent}{0em}
\addtolength{\parskip}{\baselineskip}
 
\section*{Installera från källkod}
I SUSE Linux finns många program tillgängliga redan. En del finns klara för
installation från skivorna eller de paketförråd (eng \emph{repositories}) som
finns tillgängliga på nätet. Innan du börjar fundera på att kompilera ett program
från källkod (eng \emph{source}) bör du söka i de paketförråd som finns och i
sökmotorer som \href{http://rpm.pbone.net/}{pbone.net},
\href{http://rpmfind.net/}{rpmfind.net} ,
\href{http://www.tuxfinder.com/}{tuxfinder.com} och
\href{http://www.rpmseek.com/}{rpmseek.com} . Du kan också prova att söka på
nätet med sökmotorn Google efter programmets namn och SUSE, till exempel
\texttt{apache suse rpm} . Det är inte alltid ett krav att det skall vara just
en SUSE-rpm du måste använda för att installera programmet, utan ofta går det bra
med ett RPM-paket från till exempel Red Hat eller Mandriva också. Ibland går det
att göra om ett annat paket, från Debian(.deb) eller Slackware (.tgz), så det
fungerar i SUSE Linux med hjälp av kommandot \textbf{alien}. 
 
\textsf{VARNING!} Testa aldrig att installera främmande paket i en
driftserver! Utvärdera dem på en testmaskin först så du ser att allt fungerar och ingenting
slutar fungera.

Att installera program från källkod är inte så svårt, men det kan krångla. När du
har laddat ner källkoden är den troligen komprimerad med gzip (.gz) eller bzip2
(.bz2) och arkiverad med tar (.tar). Ett paket som distribueras så kallas för
\emph{tarball} . Du kan hitta källkod för många program på följande sidor:
\href{http://freshmeat.net/}{freshmeat.net},
\href{http://sourceforge.net/}{sourceforge.net},
\href{http://www.gnomefiles.org/}{gnomefiles.org} och
\href{http://www.kde-apps.org/}{kde-apps.org}. Vi har valt att ladda ner den
påhittade filen joonix-1.2.tar.gz och skall extrahera källkoden så vi kan börja
jobba med den. Vi använder kommandot \textbf{tar} för att extrahera filerna från
arkivet, eftersom arkivet är komprimerat med \textbf{gzip} (det ser vi på
filändelsen .gz ) måste vi packa upp arkivet först:
\begin{verbatim}
$ gunzip joonix-1.2.tar.gz
\end{verbatim}
Kvar blev en fil som heter \texttt{joonix-1.2.tar} , den extraherar vi med
kommandot \textbf{tar} :
\begin{verbatim}
$ tar xf joonix-1.2.tar
\end{verbatim}
När vi tittar i vår katalog nu har vi fått en ny katalog, \texttt{joonix-1.2/}
. Oftast skapar sådana här paket en ny katalog som heter samma sak som filen vi
laddade hem, men det finns undantag. Hade filen varit komprimerad med bzip2
(filändelsen \texttt{.bz2}) hade vi packat upp den med bzip2-kommandot istället
för gunzip, vi kan också använda kommandot tar för att packa upp och extrahera
filerna på en och samma gång:
\begin{verbatim}
$ tar zxf joonix-1.2.tar.gz
$ tar jxf joonix-1.2.tar.bz2
\end{verbatim}

Skillnaden är flaggan \texttt{z} respektive flaggan \texttt{j}, beroende på om
det är ett gzip- (z) eller ett bzip2- (j) komprimerat arkiv. Filnamnen på
filer som innehåller källkod heter oftast programets namn och versionsnummer
(i vårt fall heter programmet joonix och versionen är 1.2).

Nu går vi in i katalogen som skapades med kommandot \textbf{cd joonix-1.2} .
Lista filerna och se om det finns en fil som heter \texttt{README} och/eller
\texttt{INSTALL} . README- och INSTALL-filerna innehåller information om vilket
program du laddat ner, vem som skapat det och vilken licens programmet är släppt
under. Du kan ofta hitta vilka beroenden (eng \emph{dependencies}) programmet har
och hur du skall kompilera och installera programmet. Normalt förfarande för
kompilering och installation är:
\begin{verbatim}
$ ./configure
$ make
$ make install
\end{verbatim}

Den mest kritiska fasen i kompileringen är när du kör \textbf{./configure} .
configure tittar om din dator har allt som behövs för att kompilera programmet,
saknar du något får du ett felmeddelande. Det gäller att försöka lista ut vad som
saknas och sedan testa ./configure igen. Om du vet ungefär vad paketet är gjort
för: är det ett konsollprogram, ett GNOME-program eller ett KDE-program?, kan du
ofta lösa problemen genom att gå in i YaST2, Installera och ta bort program och
välja Selections och därefter välja rätt utvecklingsmiljö (GNOME Development, KDE
Development, C/C++ Compiler and Tools). I SUSE Linux heter paket som används för
att kompilera andra program något med -devel-. Om configure säger att du saknar
till exempel Qt behöver du qt3-devel paketet för att kompilera och qt3-paketet
för att kunna köra programmet sedan. C/C++ Compiler and Tools behövs normalt sätt
alltid för program som skall kompileras i SUSE Linux.

\section*{alien}
Med kommandot alien kan du konvertera Slackware (.tgz) och Debian (.deb) paket
till RPM-paket. Det enklaste sättet att göra det är att skriva:
\begin{verbatim}
$ alien -r ett.debian.paket-1.deb
\end{verbatim}
Växeln \texttt{-r} är valet för att skapa ett RPM-paket. Tänk på att du måste ha
ett paket för din processorarkitektur och att paket du försöker installera inte är beroende
av bibliotek som inte finns tillgängliga i SUSE, i så fall måste du installera
dem först. 

\section*{Övningar}
\begin{enumerate}
\item[] Leta upp ett program på någon av sidorna vi nämner ovanför och ladda hem
källkoden för det. Kompilera programmet och testa om det fungerar i ditt SUSE.
Dokumentera vilka problem som uppstod och hur du löste dem.
\item[] Använd Alien för att konvertera ett debianpaket, du hittar dem på
\href{http://packages.debian.org/stable/}{packages.debian.org}, och installera
det i SUSE. Dokumentera vilka problem som uppstod och hur du löste dem.
\end{enumerate}


\end{document}
